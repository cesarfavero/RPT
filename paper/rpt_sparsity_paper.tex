\documentclass{article}

% NeurIPS-style formatting
\usepackage[final]{neurips_2024}
\usepackage[utf8]{inputenc}
\usepackage[T1]{fontenc}
\usepackage{hyperref}
\usepackage{url}
\usepackage{booktabs}
\usepackage{amsfonts}
\usepackage{amsmath}
\usepackage{nicefrac}
\usepackage{microtype}
\usepackage{graphicx}
\usepackage{xcolor}
\usepackage{algorithm}
\usepackage{algorithmic}
\usepackage{listings}

\lstset{
  basicstyle=\small\ttfamily,
  keywordstyle=\color{blue},
  commentstyle=\color{gray},
  breaklines=true,
  frame=single,
  language=Python
}

\title{Sparsity Improves Ternary Language Models:\\Evidence from BitNet b1.58}

\author{
  Cesar Favero\thanks{Corresponding author} \\
  Caesars Technology \\
  \texttt{cesar.favsouza@gmail.com} \\
  \texttt{cesar@caesarstechnology.com} \\
}

\begin{document}

\maketitle

\begin{abstract}
We demonstrate that magnitude-based pruning \textbf{improves} the quality of ternary language models, contrary to the conventional expectation that removing parameters degrades performance. Working with Microsoft's BitNet b1.58-2B-4T (2.4B parameters, 100\% ternary weights \{-1, 0, +1\}), we show that removing 10\% of weights by magnitude reduces perplexity by 26.1\% \textit{without any fine-tuning}. With Quantization-Aware Training using the Straight-Through Estimator (QAT/STE), we achieve a 34.8\% perplexity improvement (25.13 $\to$ 16.39 on WikiText-2) while maintaining 100\% ternary weights and 42.6\% sparsity. The resulting model generates coherent text when deployed via bitnet.cpp on CPU. We additionally show that the model naturally operates at the edge of chaos (Lyapunov exponent $\approx$ 0), consistent with Self-Organized Criticality theory, and discuss connections to thermodynamic principles of computation. Model weights and GGUF files are publicly available at \url{https://huggingface.co/CesarFavero/rpt-bitnet-2b-pruned} and \url{https://huggingface.co/CesarFavero/rpt-bitnet-2b-pruned-GGUF}.
\end{abstract}

% ============================================================
\section{Introduction}
% ============================================================

Ternary neural networks, where weights are constrained to \{-1, 0, +1\}, have emerged as a promising approach to efficient inference. Microsoft's BitNet b1.58 \citep{ma2024era} demonstrated that ternary models can match the quality of full-precision models at scale, with the subsequent release of BitNet b1.58-2B-4T---a 2.4 billion parameter ternary model trained from scratch that generates coherent text and achieves competitive benchmarks \citep{microsoft2025bitnet}.

A natural question arises: \textit{can ternary models be further compressed through sparsity?} The Lottery Ticket Hypothesis \citep{frankle2019lottery} established that dense networks contain sparse subnetworks achieving comparable accuracy. However, this was demonstrated for full-precision networks. Whether the same principle applies to already-quantized ternary networks---which operate in a fundamentally different regime with only 1.58 bits per weight---is an open question.

We present three main contributions:

\begin{enumerate}
    \item \textbf{Sparsity improves ternary models}: Removing 10\% of weights by magnitude from BitNet 2B \textit{improves} perplexity by 26.1\% without fine-tuning (Section~\ref{sec:experiments}).
    \item \textbf{QAT/STE enables functional deployment}: A complete pipeline (prune $\to$ QAT/STE $\to$ ternary snap $\to$ GGUF $\to$ CPU) achieves 34.8\% PPL improvement with coherent text generation (Section~\ref{sec:method}).
    \item \textbf{Natural criticality}: The model operates at the edge of chaos (Lyapunov $\approx$ 0) without explicit regularization, consistent with Self-Organized Criticality theory (Section~\ref{sec:analysis}).
\end{enumerate}

% ============================================================
\section{Background}
\label{sec:background}
% ============================================================

\subsection{BitNet b1.58}

BitNet b1.58 \citep{ma2024era} constrains all linear layer weights to ternary values \{-1, 0, +1\} using 1.58 bits per weight ($\log_2 3$). Each weight group has a learned scale factor:
\begin{equation}
    \hat{w} = \text{scale} \cdot \text{round}\left(\frac{w}{\text{scale}}\right)\bigg|_{\text{clamp}(-1,1)}, \quad \text{scale} = \text{mean}(|w_{\text{group}}|)
\end{equation}

The key insight is that ternary operations (additions/subtractions instead of multiplications) enable massive efficiency gains: 2.37--6.17$\times$ speedup on x86 CPUs with 71.9--82.2\% energy reduction \citep{microsoft2025bitnet}.

\subsection{Pruning and the Lottery Ticket Hypothesis}

The Lottery Ticket Hypothesis \citep{frankle2019lottery} demonstrates that dense networks contain sparse subnetworks (``winning tickets'') of 10--20\% the original size that achieve comparable accuracy when retrained from initialization. Magnitude pruning \citep{han2015learning}---removing weights with the smallest absolute values---is the simplest and most widely-used pruning strategy.

\textbf{Our setting differs from LTH}: we prune \textit{already-trained} ternary weights by magnitude without resetting to initialization. The improvement we observe suggests removal of actively harmful weights, not discovery of a subarchitecture.

\subsection{Quantization-Aware Training with STE}

QAT \citep{jacob2018quantization} simulates quantization during the forward pass while maintaining full-precision weights for gradient updates. The Straight-Through Estimator (STE) \citep{bengio2013estimating} approximates the gradient through the non-differentiable rounding operation:
\begin{equation}
    \frac{\partial \text{round}(x)}{\partial x} \approx 1
\end{equation}

While STE is a biased gradient estimator, it has proven empirically effective for quantization-aware training \citep{jacob2018quantization}.

\subsection{Self-Organized Criticality}

Self-Organized Criticality (SOC) \citep{bak1987self} describes systems that naturally evolve to the boundary between order and chaos. Biological neural networks exhibit SOC through neuronal avalanches with power-law distributions \citep{beggs2003neuronal}. At criticality, information transmission and dynamic range are maximized \citep{bertschinger2004real}.

% ============================================================
\section{Method}
\label{sec:method}
% ============================================================

Our pipeline consists of three stages: progressive pruning, QAT/STE fine-tuning, and ternary snap.

\subsection{Progressive Magnitude Pruning}

Given a ternary model with weights $W \in \{-1, 0, +1\}^{d_\text{out} \times d_\text{in}}$ and per-group scales $s \in \mathbb{R}^{G}$, we compute effective magnitudes $|s \cdot w|$ and prune globally by percentile. Pruning is applied progressively: first 5\%, then an additional 5\% of the remaining non-zero weights.

Embedding and language model head layers are protected from pruning. The pruning mask $M$ is binary and fixed after creation.

\subsection{QAT/STE Fine-tuning}

After pruning, we fine-tune with QAT/STE to recover quality while maintaining ternary compatibility:

\begin{algorithm}[H]
\caption{QAT/STE Fine-tuning for Ternary Models}
\begin{algorithmic}[1]
\REQUIRE Model $f_\theta$, masks $M$, original scales $s_{\text{orig}}$, data $\mathcal{D}$
\FOR{step $= 1$ to $N$}
    \STATE Sample batch $B \sim \mathcal{D}$
    \STATE \textbf{// Forward: quantize to ternary}
    \FOR{each weight matrix $W_l$ with mask $M_l$}
        \STATE $W_{\text{saved}} \leftarrow W_l$ \hfill \textit{(save continuous)}
        \STATE $W_{\text{quant}} \leftarrow s_{\text{orig}} \cdot \text{round}(W_l / s_{\text{orig}}) \big|_{[-1,1]}$
        \STATE $W_l \leftarrow W_{\text{quant}} \odot M_l$ \hfill \textit{(apply mask)}
    \ENDFOR
    \STATE Compute loss: $\mathcal{L} = -\sum_t \log p(x_t | x_{<t})$
    \STATE \textbf{// Backward: STE, gradients flow through continuous weights}
    \STATE $\nabla_\theta \mathcal{L}$ via backpropagation (STE: $\frac{\partial \text{round}}{\partial x} \approx 1$)
    \STATE \textbf{// Restore and update continuous weights}
    \FOR{each $W_l$}
        \STATE $W_l \leftarrow W_{\text{saved}} - \eta \cdot \nabla_{W_l} \mathcal{L}$
        \STATE $W_l \leftarrow W_l \odot M_l$ \hfill \textit{(re-apply mask)}
    \ENDFOR
\ENDFOR
\end{algorithmic}
\end{algorithm}

The key insight: the model ``sees'' ternary weights during the forward pass but has continuous weights ``behind the scenes'' for learning. This forces the model to discover representations that work with \{-1, 0, +1\}.

\subsection{Ternary Snap and Deployment}

After fine-tuning, we snap all weights to ternary:
\begin{equation}
    w_{\text{final}} = s_{\text{orig}} \cdot \text{round}(w / s_{\text{orig}}) \big|_{[-1,1]} \cdot m
\end{equation}
where $m \in \{0, 1\}$ is the pruning mask. The model is then converted to GGUF I2\_S format for CPU inference via bitnet.cpp.

\subsection{Why AdamW Without STE Fails}

We found that fine-tuning with standard AdamW (without STE) fails catastrophically at deployment. AdamW moves weights from ternary values to continuous values (e.g., $1.0 \to 0.87$). The model learns to exploit these continuous values, achieving excellent PyTorch PPL (15.43), but when re-quantized to ternary for GGUF deployment, the representations are destroyed. STE prevents this by maintaining ternary constraints during the forward pass.

% ============================================================
\section{Experiments}
\label{sec:experiments}
% ============================================================

\subsection{Setup}

\textbf{Model}: Microsoft BitNet b1.58-2B-4T-bf16 (2.4B parameters, 30 layers, hidden dim 2560, 100\% ternary). \textbf{Dataset}: WikiText-2 (train for fine-tuning, validation for PPL). \textbf{Hardware}: NVIDIA T4 (inference), H100 (progressive sparsity), A100 (QAT/STE deploy). \textbf{Training}: AdamW, lr=$5\times10^{-4}$, weight decay 0.01, cosine scheduler, 300 steps per pruning level.

\textbf{Baseline comparability note}: Baseline PPL differs across environments due precision/runtime settings (T4: 9.39; H100 with compile/TF32: 25.10; A100 without compile: 25.13). Therefore, all reported gains are computed against the baseline from the same experiment environment.

\subsection{Experiment 1: Raw Pruning (No Fine-tuning)}

Table~\ref{tab:raw_pruning} shows results of magnitude pruning without any fine-tuning, evaluated on a T4 GPU.

\begin{table}[h]
\centering
\caption{Raw magnitude pruning on BitNet 2B (T4, no fine-tuning)}
\label{tab:raw_pruning}
\begin{tabular}{lccc}
\toprule
\textbf{Sparsity} & \textbf{PPL} & \textbf{vs Baseline} & \textbf{Text Quality} \\
\midrule
0\% (baseline) & 9.39 & --- & Correct \\
10\% & \textbf{6.94} & \textbf{-26.1\%} & Correct \\
20\% & 7.88 & -16.1\% & Partial \\
30\% & 12.68 & +35.0\% & Degraded \\
\bottomrule
\end{tabular}
\end{table}

The 26.1\% improvement at 10\% sparsity \textit{without any fine-tuning} suggests that approximately 10\% of ternary weights are actively harmful.

\subsection{Experiment 2: Progressive Pruning with Fine-tuning}

Table~\ref{tab:progressive} shows progressive pruning with AdamW fine-tuning (H100, 300 steps/level). Note: this experiment used standard AdamW without STE; the resulting model has good PyTorch PPL but does not deploy correctly to GGUF (see Section~\ref{sec:method}).

\begin{table}[h]
\centering
\caption{Progressive pruning + AdamW fine-tuning (H100, 300 steps/level)}
\label{tab:progressive}
\begin{tabular}{lcccc}
\toprule
\textbf{Sparsity} & \textbf{PPL Before FT} & \textbf{PPL After FT} & \textbf{vs Baseline} & \textbf{Text} \\
\midrule
0\% (baseline) & 25.10 & 25.10 & --- & Correct \\
5\% & 24.90 & 15.05 & -40.0\% & Correct \\
10\% & 17.42 & \textbf{14.97} & \textbf{-40.4\%} & Correct \\
15\% & 18.44 & 15.36 & -38.8\% & Correct \\
20\% & 17.48 & 15.46 & -38.4\% & Correct \\
25\% & 18.08 & 16.09 & -35.9\% & Correct \\
30\% & 18.62 & 16.46 & -34.4\% & Correct \\
40\% & 30.94 & 17.73 & -29.3\% & Correct \\
50\% & 44.04 & 20.23 & -19.4\% & Repetitive \\
\bottomrule
\end{tabular}
\end{table}

\textbf{All sparsity levels from 5\% to 50\% improve over baseline after fine-tuning.} The sweet spot is 10\%, with text remaining correct up to 40\%.

\subsection{Experiment 3: QAT/STE Deploy Pipeline}

Table~\ref{tab:deploy} shows the complete deployment pipeline with QAT/STE on A100.

\begin{table}[h]
\centering
\caption{QAT/STE deploy pipeline (A100, progressive 5\%$\to$10\%)}
\label{tab:deploy}
\begin{tabular}{lc}
\toprule
\textbf{Metric} & \textbf{Value} \\
\midrule
Baseline PPL & 25.13 \\
PPL after ternary snap & \textbf{16.39 (-34.8\%)} \\
Ternary & 100\% \\
Sparsity & 42.6\% \\
GGUF size (I2\_S) & $\sim$1.1 GB \\
CPU inference & Coherent text \\
CPU speed (Colab CPU) & 0.26 tok/s \\
Training time & $\sim$7 min (A100) \\
\bottomrule
\end{tabular}
\end{table}

The reported 42.6\% sparsity includes both native zero weights already present in BitNet b1.58 and additional zeros introduced by progressive pruning.
The reported CPU speed was measured on a free Colab CPU and is hardware/runtime dependent; optimized x86/ARM implementations can be substantially faster.

Notably, PPL \textit{improved} after the ternary snap (from 33.07 during QAT to 16.39 after snap). We hypothesize this reflects implicit regularization: the snap removes small continuous deviations accumulated during STE training that were noise rather than signal.

\subsection{Experiment 4: Activation Pruning (Negative Result)}

Motivated by predictive coding theory \citep{rao1999predictive}, we tested whether \textit{activations} (not weights) could also be pruned. Results were negative:

\begin{table}[h]
\centering
\caption{Activation pruning by percentile (T4)}
\label{tab:activation}
\begin{tabular}{lccc}
\toprule
\textbf{Percentile Zeroed} & \textbf{PPL} & \textbf{vs Baseline} \\
\midrule
0\% (baseline) & 25.09 & --- \\
P50 & 40.26 & +60.5\% \\
P60 & 96.64 & +285\% \\
P70 & 318.85 & +1171\% \\
\bottomrule
\end{tabular}
\end{table}

\textbf{Contrast}: 10\% weight pruning improves PPL by 26\%; 50\% activation pruning degrades it by 60\%. Weights encode learned \textit{structure} (many are redundant); activations encode \textit{information} (nearly none are redundant).

% ============================================================
\section{Analysis}
\label{sec:analysis}
% ============================================================

\subsection{Self-Organized Criticality}

We measured the propagation of small perturbations through the model's 30 layers by adding Gaussian noise ($\varepsilon = 0.01$, relative to embedding norm) and tracking the relative deviation at each layer:
\begin{equation}
    \delta_l = \frac{\|h_l^{\text{perturbed}} - h_l^{\text{original}}\|}{\|h_l^{\text{original}}\|}
\end{equation}

The Lyapunov exponent, estimated as the average logarithmic growth rate $\lambda = \frac{1}{L}\sum_l \ln(\delta_{l+1}/\delta_l)$, was $\lambda = -0.002 \approx 0$, indicating operation at the \textbf{edge of chaos}. The total amplification across 30 layers was $0.94\times \approx 1.0$, with a characteristic ``bell'' shape: perturbation grows until layers 13--14 (network center) then decreases back, suggesting an active self-correction mechanism.

\subsection{Branching Ratio Analysis}

The branching ratio (output norm / input norm per layer) converges toward 1.0 with depth: from 1.70 (Layer 1) to 1.09 (Layer 29), excluding Layer 0 which is an outlier (40.04, embedding transformation). This convergence toward unity is a hallmark of critical systems \citep{beggs2003neuronal}.

\subsection{Why Does Sparsity Help?}

We propose three complementary explanations:

\textbf{Implicit regularization}: In the ternary regime, weights can only be \{-1, 0, +1\}. Low-magnitude weights (those closest to the pruning threshold) carry the least information. Removing them reduces model complexity, acting as regularization. This is consistent with the improvement being most pronounced at moderate sparsity (10--20\%).

\textbf{Noise removal}: During ternary training, some weights may settle at non-zero values that are artifacts of the optimization trajectory rather than meaningful representations. Magnitude pruning removes these ``accidental'' non-zero weights.

\textbf{Thermodynamic interpretation}: From a Landauer principle perspective \citep{landauer1961irreversibility}, each non-zero weight represents information that must be processed. Removing unnecessary information reduces the computational cost without reducing (and potentially improving) the information content of the output. This is analogous to the brain's synaptic pruning during development \citep{huttenlocher1979synaptic}.

\subsection{Connection to RPT Framework}

These results are consistent with the RPT (Redes Preditivas Termodinamicas) framework, which proposes that efficient neural computation should obey thermodynamic principles:

\begin{itemize}
    \item \textbf{Landauer's principle} $\to$ Sparsity reduces information processing cost
    \item \textbf{Self-Organized Criticality} $\to$ Model naturally operates at edge of chaos
    \item \textbf{Predictive Coding} $\to$ Correction ratios decrease with depth (39.87 $\to$ 0.21)
\end{itemize}

The framework additionally proposes Equilibrium Propagation (local learning without backpropagation) and Holographic Attention ($O(n \cdot R)$ instead of $O(n^2)$), which remain to be validated empirically.

% ============================================================
\section{Related Work}
\label{sec:related}
% ============================================================

\textbf{Ternary/Binary Networks}: \citet{ma2024era} introduced BitNet b1.58 with ternary weights. \citet{zhu2024scalable} achieved MatMul-free inference with ternary weights on FPGA at 13W. BinaryConnect \citep{courbariaux2015binaryconnect} and XNOR-Net \citep{rastegari2016xnornet} explored binary weights.

\textbf{Pruning}: The Lottery Ticket Hypothesis \citep{frankle2019lottery} showed dense networks contain sparse subnetworks. SparseGPT \citep{frantar2023sparsegpt} enables one-shot pruning of large models. Wanda \citep{sun2023simple} prunes by weight $\times$ activation magnitude.

\textbf{Criticality in Neural Networks}: \citet{bertschinger2004real} showed maximum computational capacity at the edge of chaos. \citet{ramsauer2021hopfield} connected Transformer attention to Hopfield network dynamics. \citet{bahri2020statistical} applied statistical mechanics to deep learning.

\textbf{Efficient Inference}: bitnet.cpp \citep{microsoft2025bitnet} demonstrates 100B-parameter models on single CPU. GPTQ \citep{frantar2022gptq} and AWQ \citep{lin2024awq} enable post-training quantization. GGUF format \citep{ggml2023} standardizes quantized model deployment.

% ============================================================
\section{Limitations and Future Work}
\label{sec:limitations}
% ============================================================

\textbf{Evaluation scope}: Our results are on WikiText-2 only. Validation on MMLU, HellaSwag, ARC, and other standard benchmarks is needed to confirm the PPL improvement translates to actual capability.

\textbf{PPL improvement caveat}: The dramatic improvement after ternary snap (33.07 $\to$ 16.39) may reflect overfitting to WikiText-2 or implicit regularization specific to this dataset. Broader evaluation is critical.

\textbf{Scale}: We tested only one model (BitNet 2B). Whether sparsity consistently improves ternary models across scales (7B, 13B, 70B) is an open question.

\textbf{Sparsity levels}: Only progressive 5\%$\to$10\% was tested with QAT/STE. Higher sparsity levels (20--40\%) with STE may yield even better results.

\textbf{Artifact scope}: At the time of writing, the public release includes model weights and GGUF artifacts. A public code repository with training/deployment scripts is not yet available.

\textbf{Future work}: (1) Rigorous benchmarks (MMLU, HellaSwag, ARC); (2) Higher sparsity with QAT/STE; (3) Apply QAT/STE to full-precision models to test generic ternary conversion; (4) Equilibrium Propagation at scale; (5) Structured 2:4 sparsity for hardware acceleration.

% ============================================================
\section{Conclusion}
\label{sec:conclusion}
% ============================================================

We demonstrated that magnitude-based pruning improves ternary language models, achieving a 34.8\% perplexity improvement on WikiText-2 with a complete deployment pipeline running on CPU. The model naturally operates at the edge of chaos (Lyapunov $\approx$ 0), connecting to Self-Organized Criticality theory. These results suggest that ternary models are significantly over-parameterized and that thermodynamic principles---specifically Landauer's principle and SOC---provide useful predictive frameworks for neural network efficiency.

\textbf{Reproducibility}: Public artifacts include model weights and GGUF files at \url{https://huggingface.co/CesarFavero/rpt-bitnet-2b-pruned} (model) and \url{https://huggingface.co/CesarFavero/rpt-bitnet-2b-pruned-GGUF} (GGUF).

% ============================================================
\section*{Acknowledgments}
% ============================================================

We thank Microsoft for releasing BitNet b1.58-2B-4T under MIT license, the ggml/llama.cpp community for the inference framework, and Lightning AI for A100 compute access.

% ============================================================
\bibliographystyle{plainnat}

\begin{thebibliography}{30}

\bibitem[Bak(1987)]{bak1987self}
Per Bak.
\newblock Self-organized criticality: An explanation of the 1/f noise.
\newblock \emph{Physical Review Letters}, 59(4):381, 1987.

\bibitem[Bahri et~al.(2020)]{bahri2020statistical}
Yasaman Bahri, Jonathan Kadmon, Jeffrey Pennington, Sam~S Schoenholz, Jascha Sohl-Dickstein, and Surya Ganguli.
\newblock Statistical mechanics of deep learning.
\newblock \emph{Annual Review of Condensed Matter Physics}, 11:501--528, 2020.

\bibitem[Beggs and Plenz(2003)]{beggs2003neuronal}
John~M Beggs and Dietmar Plenz.
\newblock Neuronal avalanches in neocortical circuits.
\newblock \emph{Journal of Neuroscience}, 23(35):11167--11177, 2003.

\bibitem[Bengio et~al.(2013)]{bengio2013estimating}
Yoshua Bengio, Nicholas L{\'e}onard, and Aaron Courville.
\newblock Estimating or propagating gradients through stochastic neurons for conditional computation.
\newblock \emph{arXiv:1308.3432}, 2013.

\bibitem[Bertschinger and Natschl{\"a}ger(2004)]{bertschinger2004real}
Nils Bertschinger and Thomas Natschl{\"a}ger.
\newblock Real-time computation at the edge of chaos in recurrent neural networks.
\newblock \emph{Neural Computation}, 16(7):1413--1436, 2004.

\bibitem[Courbariaux et~al.(2015)]{courbariaux2015binaryconnect}
Matthieu Courbariaux, Yoshua Bengio, and Jean-Pierre David.
\newblock Binaryconnect: Training deep neural networks with binary weights during propagations.
\newblock \emph{NeurIPS}, 2015.

\bibitem[Frankle and Carbin(2019)]{frankle2019lottery}
Jonathan Frankle and Michael Carbin.
\newblock The lottery ticket hypothesis: Finding sparse, trainable neural networks.
\newblock \emph{ICLR}, 2019.

\bibitem[Frantar and Alistarh(2023)]{frantar2023sparsegpt}
Elias Frantar and Dan Alistarh.
\newblock SparseGPT: Massive language models can be accurately pruned in one-shot.
\newblock \emph{ICML}, 2023.

\bibitem[Frantar et~al.(2022)]{frantar2022gptq}
Elias Frantar, Saleh Ashkboos, Torsten Hoefler, and Dan Alistarh.
\newblock GPTQ: Accurate post-training quantization for generative pre-trained transformers.
\newblock \emph{arXiv:2210.17323}, 2022.

\bibitem[Han et~al.(2015)]{han2015learning}
Song Han, Jeff Pool, John Tran, and William~J Dally.
\newblock Learning both weights and connections for efficient neural networks.
\newblock \emph{NeurIPS}, 2015.

\bibitem[Huttenlocher(1979)]{huttenlocher1979synaptic}
Peter~R Huttenlocher.
\newblock Synaptic density in human frontal cortex—developmental changes and effects of aging.
\newblock \emph{Brain Research}, 163(2):195--205, 1979.

\bibitem[Jacob et~al.(2018)]{jacob2018quantization}
Benoit Jacob, Skirmantas Kligys, Bo~Chen, Menglong Zhu, Matthew Tang, Andrew Howard, Hartwig Adam, and Dmitry Kalenichenko.
\newblock Quantization and training of neural networks for efficient integer-arithmetic-only inference.
\newblock \emph{CVPR}, 2018.

\bibitem[Landauer(1961)]{landauer1961irreversibility}
Rolf Landauer.
\newblock Irreversibility and heat generation in the computing process.
\newblock \emph{IBM Journal of Research and Development}, 5(3):183--191, 1961.

\bibitem[Lin et~al.(2024)]{lin2024awq}
Ji~Lin, Jiaming Tang, Haotian Tang, Shang Yang, Xingyu Dang, Chuang Gan, and Song Han.
\newblock AWQ: Activation-aware weight quantization for on-device LLM compression and acceleration.
\newblock \emph{MLSys}, 2024.

\bibitem[Ma et~al.(2024)]{ma2024era}
Shuming Ma, Hongyu Wang, Lingxiao Ma, Lei Wang, Wenhui Wang, Shaohan Huang, Li~Dong, Ruiping Wang, Jilong Xue, and Furu Wei.
\newblock The era of 1-bit LLMs: All large language models are in 1.58 bits.
\newblock \emph{arXiv:2402.17764}, 2024.

\bibitem[Microsoft(2025)]{microsoft2025bitnet}
Microsoft.
\newblock BitNet b1.58-2B-4T: 2 billion parameter ternary model.
\newblock \url{https://huggingface.co/microsoft/bitnet-b1.58-2B-4T-bf16}, 2025.

\bibitem[ggml-org(2023)]{ggml2023}
ggml-org.
\newblock GGUF file format and llama.cpp tooling.
\newblock \url{https://github.com/ggml-org/llama.cpp}, 2023.

\bibitem[Ramsauer et~al.(2021)]{ramsauer2021hopfield}
Hubert Ramsauer, Bernhard Sch{\"a}fl, Johannes Lehner, Philipp Seidl, Michael Widrich, Thomas Adler, Lukas Gruber, Markus Holzleitner, Milena Pavlovi{\'c}, Geir~Kjetil Sandve, et~al.
\newblock Hopfield networks is all you need.
\newblock \emph{ICLR}, 2021.

\bibitem[Rao and Ballard(1999)]{rao1999predictive}
Rajesh~PN Rao and Dana~H Ballard.
\newblock Predictive coding in the visual cortex: a functional interpretation of some extra-classical receptive-field effects.
\newblock \emph{Nature Neuroscience}, 2(1):79--87, 1999.

\bibitem[Rastegari et~al.(2016)]{rastegari2016xnornet}
Mohammad Rastegari, Vicente Ordonez, Joseph Redmon, and Ali Farhadi.
\newblock XNOR-Net: Imagenet classification using binary convolutional neural networks.
\newblock \emph{ECCV}, 2016.

\bibitem[Sun et~al.(2023)]{sun2023simple}
Mingjie Sun, Zhuang Liu, Anna Bair, and J~Zico Kolter.
\newblock A simple and effective pruning approach for large language models.
\newblock \emph{arXiv:2306.11695}, 2023.

\bibitem[Zhu et~al.(2024)]{zhu2024scalable}
Rui-Jie Zhu, Yu~Zhang, Ethan Sifferman, Tyler Sheaves, Yiqiao Wang, Dustin Richmond, Peng Zhou, and Jason~K Eshraghian.
\newblock Scalable MatMul-free language modeling.
\newblock \emph{arXiv:2406.02528}, 2024.

\end{thebibliography}

% ============================================================
\appendix
\section{RPT Framework Overview}
\label{app:rpt}

RPT (Redes Preditivas Termodinamicas) is a theoretical framework proposing that efficient neural computation should follow five physics-inspired principles:

\begin{enumerate}
    \item \textbf{Free Energy Principle} (Friston, 2010): Process only prediction errors, not all information
    \item \textbf{Landauer's Principle} (1961): Sparsity reduces computational energy cost --- \textbf{VALIDATED}
    \item \textbf{Self-Organized Criticality} (Bak, 1987; Beggs \& Plenz, 2003): Operate at edge of chaos --- \textbf{VALIDATED}
    \item \textbf{Equilibrium Propagation} (Scellier \& Bengio, 2017): Local learning without backpropagation --- Not yet tested
    \item \textbf{Holographic Principle} (Susskind, 1995): $O(n \cdot R)$ attention via boundary encoding --- Not yet tested
\end{enumerate}

Three of five principles have been validated empirically on BitNet b1.58-2B-4T.

\section{Key Code Snippets}
\label{app:code}

\subsection{STE Forward Pass}
\begin{lstlisting}
# Forward: quantize to ternary
w_saved = param.data.clone()
w = param.data.float()
nonzero = w[w != 0]
scale = original_scales[name]
codes = (w / scale).round().clamp(-1, 1)
param.data = (codes * scale).to(param.dtype)

# After loss.backward():
param.data = w_saved - lr * param.grad  # STE
param.data *= mask  # maintain pruning
\end{lstlisting}

\subsection{Magnitude Pruning}
\begin{lstlisting}
# Collect magnitudes from all layers
for name, param in model.named_parameters():
    if param.dim() >= 2:
        w = param.data.float()
        nonzero = w[w != 0].abs()
        samples.append(nonzero)

# Global threshold
all_mags = torch.cat(samples)
threshold = all_mags.sort().values[
    int(len(all_mags) * sparsity / 100)
]

# Apply mask
mask = param.data.abs() > threshold
param.data *= mask.float()
\end{lstlisting}

\subsection{Ternary Snap}
\begin{lstlisting}
for name, param in model.named_parameters():
    if param.dim() >= 2:
        w = param.data.float()
        scale = original_scales[name]
        codes = (w / scale).round().clamp(-1, 1)
        param.data = (codes * scale).to(dtype)
\end{lstlisting}

\section{Deployment Bug Log}
\label{app:bugs}

We encountered and resolved 10 bugs during HuggingFace $\to$ GGUF $\to$ bitnet.cpp conversion. The most critical was Bug \#10: the converter wrote architecture name ``bitnet'' instead of ``bitnet-b1.58'', triggering the wrong computational graph in llama.cpp (\texttt{build\_bitnet()} vs \texttt{build\_bitnet\_158()}). This produced plausible but incoherent text with no error messages, requiring $\sim$6 hours to diagnose. The full bug list is available in our documentation.

\end{document}
